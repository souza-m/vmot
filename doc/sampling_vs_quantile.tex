%% LyX 2.3.6 created this file.  For more info, see http://www.lyx.org/.
%% Do not edit unless you really know what you are doing.
\documentclass[english]{article}
\usepackage[T1]{fontenc}
\usepackage[latin9]{luainputenc}
\usepackage{units}
\usepackage{amsmath}
\usepackage{babel}
\begin{document}

\paragraph{Discussion: sampling vs. quantile approach.}

We describe the numerical penalization method being currently used,
with emphasis on the sampling of points for penalization and discuss
an alternative option to consider. To approximate a solution of equation
2.6 with constraint 2.7, we want to minimize 
\[
\sum_{i}\int\phi_{i}\mu_{i}+\sum_{i}\int\psi_{i}d\nu_{i}+\int b_{\gamma}\left(c-\varphi\right)d\theta
\]
 where $b_{\gamma}$ is a penalization function defined as
\[
b_{\gamma}\left(t\right)=\frac{1}{\gamma}\frac{1}{2}\left[\left(\gamma t\right)^{+}\right]^{2}
\]

Computationally, each function $\phi_{i},\psi_{i},h_{i}$ is implemented
as a neural network that takes any floating-point number ($\phi_{i},\psi_{i}$)
or vector ($h_{i}$) as input. The sum of integrals above becomes
a discrete sum (to be detailed later), and is referenced to a gradient
descent algorithm as a loss function (that is, the one to be minimized
numerically). The proposed alternative is to work on the quantile
domain. Let $F_{i},G_{i}$ be the cumulative distribution functions
of $x_{i},y_{i}$ (assumed to be known) and define
\begin{alignat*}{1}
\hat{x}_{i} & =F\left(x_{i}\right)\\
\hat{y}_{i} & =G\left(y_{i}\right)\\
\hat{\phi}_{i}\left(\hat{y_{i}}\right) & =\phi_{i}\left(y_{i}\right)\\
\hat{\psi_{i}}\left(\hat{y_{i}}\right) & =\psi_{i}\left(y_{i}\right)\\
\hat{h}_{i}\left(\hat{x}\right) & =h_{i}\left(x\right)
\end{alignat*}

Now each $\hat{\phi}_{i},\hat{\psi}_{i},\hat{h}_{i}$ is implemented
by a neural network instead of their primitives. Instead of random
sampling, we use a grid of quantiles. Our marginals are uniform discrete
probabilities depending on the choice of grid size. The sampling measure
$\hat{\theta}$ is a discrete coupling of uniform discrete marginals.
Some advantages of this ``quantile'' approach are
\begin{itemize}
\item It spreads the area covered by the training evenly. This is critical
in the high dimension case where much of the area can be left uncovered
with random sampling.
\item The neural network will potentially perform better in the interval
$\left[0,1\right]$ than in the line. A good reason to believe that
is the tail noise observed in the normal-marginals example. To be
checked.
\item Potential to find ways to update of the sampling measure $\hat{\theta}$.
\end{itemize}

\subparagraph*{Approximations to the penalization integral: description and comparison.}

We approximate the penalty term $\varepsilon=\int b_{\gamma}\left(c-\varphi\right)d\theta$
with a sum, as described below.
\begin{enumerate}
\item Sampling. Let $\mathcal{X}_{i}=\left\{ x_{i}^{1},\ldots x_{i}^{n}\right\} ,\mathcal{Y}_{i}=\left\{ y_{i}^{1},\ldots y_{i}^{n}\right\} $
be (simulated) samples, $i=1,\ldots,d$. Note 1: the marginal probabilities
are expressed in the samples; their construction is the only place
where the marginal probabilities are ever considered. Note 2: In the
empirical example, even though we are using actual market data, the
method first builds marginal distributions of random, future prices
and then generates a sample from those distributions. Hence, the final
sample is simulated in any case. 
\begin{itemize}
\item Current method.

First step is to build a common sample $\Theta_{1}$ of $2d$-dimension
points. In the independent case, we wrap the points arbitrarily as
(for instance) 
\[
\Theta_{1}=\left\{ \left(x_{1}^{1},\ldots,x_{d}^{1},y_{1}^{1},\ldots,y_{d}^{1}\right),\ldots,\left(x_{1}^{n},\ldots,x_{d}^{n},y_{1}^{n},\ldots,y_{d}^{n}\right)\right\} 
\]

In the monotone coupling case, the $x_{i}^{k}$'s are first ordered
as $x_{i}^{\left(1\right)},\ldots,x_{i}^{\left(n\right)}$ and then
coupled in order, while the $y_{i}^{k}$'s are coupled arbitrarily,
like 
\[
\Theta_{1}=\left\{ \left(x_{1}^{\left(1\right)},\ldots,x_{d}^{\left(1\right)},y_{1}^{1},\ldots,y_{d}^{1}\right),\ldots,\left(x_{1}^{\left(n\right)},\ldots,x_{d}^{\left(n\right)},y_{1}^{n},\ldots,y_{d}^{n}\right)\right\} 
\]

The penalty is given by
\[
\varepsilon_{1}=\frac{1}{\left|\Theta_{1}\right|}\sum_{\left(x,y\right)\in\Theta_{1}}b_{\gamma}\left(c\left(x,y\right)-\varphi\left(x,y\right)\right)
\]

\item Separated sampling. The common sample $\Theta_{2}$ is the product
of the marginal samples. In the independent coupling case, the penalty
is calculated as
\[
\varepsilon_{2}=\frac{1}{\prod_{i}\prod_{j}\left|\mathcal{X}_{i}\right|\left|Y_{i}\right|}\sum_{x_{1}\in\mathcal{X}_{1}}\ldots\sum_{x_{d}\in\mathcal{X}_{d}}\sum_{y_{1}\in\mathcal{Y}_{1}}\ldots\sum_{y_{d}\in\mathcal{Y}_{d}}b_{\gamma}\left(c\left(x,y\right)-\varphi\left(x,y\right)\right)
\]
In the monotone case, the choice of $x$ can be seen as a function
of $x_{1}$, therefore we can write 
\[
\varepsilon_{2}=\frac{1}{\left|\mathcal{X}_{1}\right|\prod_{j}\left|Y_{i}\right|}\sum_{x_{1}\in\mathcal{X}_{1}}\sum_{y_{1}\in\mathcal{Y}_{1}}\ldots\sum_{y_{d}\in\mathcal{Y}_{d}}b_{\gamma}\left(c\left(x\left(x_{1}\right),y\right)-\varphi\left(x\left(x_{1}\right),y\right)\right)
\]
To keep the complexity comparable, we must have either $\prod_{i}\prod_{j}\left|\mathcal{X}_{i}\right|\left|Y_{i}\right|=n^{2d}$
(in the independent case) or $\left|\mathcal{X}_{1}\right|\prod_{j}\left|Y_{i}\right|=n^{d+1}$
(in the monotone coupling case) in the same scale as $\left|\Theta_{1}\right|$
in the previous method. Thus, suppose we used $\left|\Theta_{1}\right|=1M$
there. Here, in the independent case, we should use a comparable $n=\left(1M\right)^{\nicefrac{1}{4}}\sim32$
for $d=2$ and $n=\left(1M\right)^{\nicefrac{1}{6}}=10$ for $d=3$.
In the monotone coupling case, we should use $n=\left(1M\right)^{\nicefrac{1}{3}}=100$
for $d=2$ and $n=\left(1M\right)^{\nicefrac{1}{4}}\sim32$ for $d=3$.
\end{itemize}
Notice that, provided the sample sets have the same size, there is
no difference between the two methods just discussed in terms of computational
effort (other than possibly memory allocation). But in the separated
sampling case, given that marginal sizes are small specially in higher
dimension, we may have some areas of the marginal distributions being
scarcely covered throughout $\Theta$.
\item Quantile method. The points $\left(\hat{x},\hat{y}\right)$ are taken
from a grid in the 1-hypercube. So, for example, if the grid granularity
is $10$, we divide each marginal of the cube into 10 intervals, and
pick the middle points 
\begin{align*}
\hat{x}_{i}\in\hat{\mathcal{X}}_{i} & =\left\{ 0.05,\ldots,0.95\right\} \\
\hat{y}_{i}\in\hat{\mathcal{Y}}_{i} & =\left\{ 0.05,\ldots,0.95\right\} 
\end{align*}

The points $\left(\hat{x},\hat{y}\right)$ are attributed the discrete
probability $\hat{\theta}$ with uniform marginals. The penalty is
given by
\[
\varepsilon_{3}=\frac{1}{\prod_{i}\prod_{j}\left|\mathcal{\hat{X}}_{i}\right|\left|\hat{Y}_{i}\right|}\sum_{\hat{x}_{1}\in\hat{\mathcal{X}}_{1}}\ldots\sum_{\hat{x}_{d}\in\hat{\mathcal{X}}_{d}}\sum_{\hat{y}_{1}\in\hat{\mathcal{Y}}_{1}}\ldots\sum_{\hat{y}_{d}\in\hat{\mathcal{Y}}_{d}}\hat{\theta}\left(\hat{x},\hat{y}\right)b_{\gamma}\left(c\left(x,y\right)-\varphi\left(x,y\right)\right)
\]

Notice that the concept of sampling measure has entered the formula
through the probability $\hat{\theta}$, while previously it only
influenced the construction of the sample $\Theta$. In the simplest,
independent case, $\hat{\theta}$ is the independent coupling, which
is the uniform probability $\hat{\theta}\left(\hat{x},\hat{y}\right)=\frac{1}{n^{2d}}$.
In the positive monotone coupling case, only points satisfying $\hat{x}_{i}=\hat{x}_{j}\forall i,j$
have positive probability. So, if the $y$-side of $\hat{\theta}$
is independent, we set 
\[
\hat{\theta}\left(\hat{x},\hat{y}\right)=\begin{cases}
\frac{1}{n^{d+1}} & \hat{x}_{i}=\hat{x}_{j}\forall i,j\\
0 & \text{otherwise}
\end{cases}
\]
There could be interesting cases where additional information is available,
giving rise to a more elaborated and efficient $\hat{\theta}$. One
topic to investigate is whether it is possible to update $\hat{\theta}$
with information from $\hat{\varphi}$ in the same fashion as in formula
2.6 of Eckstein and Kupper (2021).

Note: There might be some literature about similar quantile-based
methods. In their brief introduction to Example 4.6, Eckstein and
Kupper (2021) mention the following papers, not yet analyzed.
\begin{itemize}
\item Embrechts, P., Puccetti, G., R�schendorf, L. \textquotedbl Model
uncertainty and VaR aggregation.\textquotedbl{} (2013, J. Bank. Financ.)
\item Puccetti, G., R�schendorf, L. \textquotedbl Computation of sharp
bounds on the distribution of a function of dependent risks.\textquotedbl{}
(J. Comput. Appl. Math., 2012)
\end{itemize}
\end{enumerate}

\end{document}
