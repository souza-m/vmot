%% LyX 2.3.6 created this file.  For more info, see http://www.lyx.org/.
%% Do not edit unless you really know what you are doing.
\documentclass[english]{article}
\usepackage[T1]{fontenc}
\usepackage[latin9]{luainputenc}
\usepackage{amsmath}
\usepackage{babel}
\begin{document}
(Financial motivation, could be moved to a previous section -- MS.)
We are especially interested in the simplified cross product cost
function $c\left(x,y\right)=y_{1}y_{2}$, which relates to the problem
of robust pricing of the covariance of financial assets. From an asset
pricing point of view, knowing the marginals of $X_{i}$ and $Y_{i}$,
$i=1,2$, is equivalent to knowing the future distribution of prices
of two assets at two distinct future times $t=1,2$. When their joint
distribution is unknown, VMOT provides an upper bound to the expected
covariance at $t=2$. In fact, we can derive the marginal distributions
from market data, as we showcase in our second example below. The
problem of finding the lower bound is symmetrical (when $d=2$) and
therefore not considered here. Notice that, even though the prices
at $t=1$ are not included in the cost function, we expect that knwoledge
about the distribution at $t=1$ narrows the bounds for the expected
value of the cost function, as in {[}21{]}.

\paragraph*{Example 1 (Normal marginals).}

We start with a theoretical example where the marginals are distributed
as
\begin{align*}
X_{1},X_{2} & \sim N\left(0,1\right)\\
Y_{1} & \sim N\left(0,2\right)\\
Y_{2} & \sim N\left(0,3\right)
\end{align*}

From Proposition 6.1 in the appendix, the solution is calculated as
$p^{+}=1+\sqrt{2}\approx2.4142$. We run the method with a sample
of 2 milion points drown form the independent coupling of the marginals.
(Maybe add details about the neural network here.) The following graph
shows the convergence of the dual value approximation in the two formulations,
namely, with our simplified version shown in blue and the analogous
full-dimension version shown in orange. It is interesting to notice
that both methods provide very similar results, even though the simplified
one allocates less memory and demands slightly less time to process
the same number of iterations.

(Further discussion?)

\paragraph*{Empirical example.}

Moving to a real world application, we construct implicit marginal
distributions for the prices of Apple and Amazon stocks at times Jan
20th and Feb 17th, 2023 based on call and put option prices as of
Dec 16th, 2022 (Note: update dates and prices before the last version
of the paper) using the method described at an appendix (method used
by Joshua). 

(To continue.)
\end{document}
